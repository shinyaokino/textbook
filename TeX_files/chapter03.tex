\chapter{Microsoft Word}

\section{はじめに}
ワープロは文章を作成するアプリケーションである。
文字の属性(フォント、大きさ、文字飾り)の指定、図表の取り込みなどが可能であり、レイアウトの最終形を見ながら作業を行い、印刷することを目的としている。

演習室で使用するパソコンには、ワープロソフトとして、Microsoft Officeの「Microsoft Word 2010」とLibre Officeの「Writer」がインストールされている。
Microsoft Wordは、マイクロソフトがWindows及びMac OS X向けに販売している文書作成ソフトウェアであり、表計算ソフトウェアExcelとともにMicrosoft Officeの中核をなすアプリケーションである。
一般的にMicrosoft Wordはワード(WordまたはMS-Word)と呼ばれることが多い。
ここでは、「Microsoft Word 2010」を用いて、講義・実験のレポート、セミナーのレジュメ、卒業論文などの作成に必要な「文字の修飾、配置」、「数式の入力」、「表の作成」などについて演習する。

\section{Wordの起動、終了、文章の保存}
\subsection{Wordの起動}
\begin{enumerate}
\item Windows 7のデスクトップから、[スタート]ボタン(Windowsのマークのボタン)をクリックする。
\item スタートメニューの[すべてのプログラム]ボタンをクリックする。
\item プログラムメニュー(図1)の[Microsoft Office]→[Microsoft Office Word 2010]を選択する。
\end{enumerate}

\subsection{Wordの終了}
Wordを終了するには、メニューバー左上の[ファイル]を選択し、プルダウンメニューの[終了]を選択する。
あるいは、タイトルバー右上端の×[閉じる]ボタンをクリックする。
終了の際、保存されていない文章がある場合には、確認メッセージが表示される。

\subsection{文章の保存}
\begin{enumerate}
\item [ファイル]→[名前を付けて保存]を選択する。
\item [名前を付けて保存]ダイアログボックスが開く。
\item [ファイル名]にファイル名を入力する。
\item 保存先を指定する。なお、演習室では「マイ ドキュメント(ドキュメント)」、または「マイ ドキュメント(ドキュメント)」内に作成したフォルダ以外は指定しないようにすること(図2)。「
マイドキュメント(ドキュメント)」以外に保存したファイルは、ログオフすると消去されてしまうので注意が必要である。
\item [保存]をクリックする。
\end{enumerate}

\section{文字の修飾、配置}
\subsection{文字の修飾}
日本語のフォントにはMS明朝をはじめMS P明朝、MSゴシック、MS P ゴシック、MS UI ゴシックなどがある。Pの意味はProportionalで、文字幅を詰めてくれる(「っ」と「つ」は文字幅が異なるので、それを詰めてくれる)。MS UIゴシックのカタカナはMS P ゴシックと異なり少し幅が狭い。一方、英語ではCentury, Arial, Times New Roman, Symbolなどがある。Symbolはギリシャ文字を書くときに便利である。日本語で「あるふぁ」と入力して変換すればギリシャ文字(α)を打てるが、全角となる。文章を書く際には、全角と半角に気をつけなければならない。Windowsは全角、Windowsは半角である。基本的に英文字・数字は半角、日本語文字は全角を使用すると美しくなる。
フォントを太くする(太字)、イタリックにする(斜体)、下線を付ける(下線)などの加工を行いたい場合や、Cu+やFe2O3などの「上付き」、「下付き」を使用したい場合は、[ホーム]→[フォント]にあるボタンから行う(図3)。また、フォントに関してさらに詳細な設定をしたい場合は、[ホーム]→[フォント]の右下にある矢印のボタンを用いて行うことができる(図3)。なお、科学論文中の物理量は、通常イタリックで入力することが多い。

