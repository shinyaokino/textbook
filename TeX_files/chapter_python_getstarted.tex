\chapter{Python入門}

この章では、プログラミング言語の一つであるPythonの基本的な使い方を学ぶ。
プログラミングスキルを上達させるコツは、たくさんのコードを書くことである。
以下に示すコードは全てJupyter-notebookで実行させること。





\section{\ref{sec:jupyter_example}章の内容の学習}

この章では、前章\ref{sec:jupyter_example}で入力した内容について学ぶ。
それを通して、前回とは異なるデータに対して同様の処理を行えるよう、プログラミングの基礎を学ぶ。

%TODO


\section{プログラミングの基礎}

\subsection{変数の型\label{subsec:type}}

プログラミング基礎知識の中で最も重要な概念が「変数の型」である。

「型」は変数の役割を区別するものである。

Pythonにデフォルトで用意されている型には以下のようなものがある。
\begin{itemize}
	\item {\ttfamily int} 整数を表す。
	\item {\ttfamily float} 実数を表す。
	\item {\ttfamily bool} 真・偽の2値を表す。
	\item {\ttfamily str} 文字列を表す。
	\item {\ttfamily list, dict, tuple} 複数の要素から成る変数の組を表す。
\end{itemize}

これらの役割はおいおい明らかになるとして、
まずはいろいろな変数の型を調べてみよう。

%%%%%%%%%%%%%%%%%%%%%%%%%%%%%%%%%%%%%%%%%%%%%%%%%%%%%%%

{\bfseries 課題\ref{subsec:type}}

\noindent
カッコの中の変数(引数:ひきすう と言う)の型を返す関数である{\ttfamily type()}関数を用いた以下の命令を実行せよ。
なお、ここでは、{\ttfamily print()}関数の引数に{\ttfamily type()}関数の値を代入することで、その結果を画面に表示している。


\begin{itemize}
	{\ttfamily 
		\item value\_float = 1.0

		print(type(value\_float))

		\item value\_str = 'Hello world'

		print(type(value\_str))
		
		\item value\_list = [0.1, 0.2, 3.0]
		
		print(type(value\_list))
	}
\end{itemize}

%%%%%%%%%%%%%%%%%%%%%%%%%%%%%%%%%%%%%%%%%%%%%%%%%%%%%%%

\subsubsection{int, float型\label{subsubsec:int-float}}
{\ttfamily int, float}は、数を表す型である。
int や float の例としては以下のような例がある。


{\bfseries 課題\ref{subsubsec:int-float}}

\noindent
以下の命令について、その値および型を表示させるとともに、その意味を答えよ。
必要であれば
%TODO
を参考にすること。

なお、命令の意味については、実行セルの直下にMarkdownセルを作成して記入すること。
\begin{comment}
\begin{enumerate}
	{\ttfamily 
		\item a = 2.0\\
			  b = 3.0 ** 2.0\\
		
		\item a = 5 \% 2\\
		
		\item a = 3 * 2.0 + 1.0\\
	}
\end{enumerate}
content...
\end{comment}


\subsubsection{str型\label{subsubsec:str}}
{\ttfamily str}は、文字列を表す型である。
文字列は以下のように用いることができる。

{\bfseries 課題\ref{subsubsec:str}}
以下の命令を実行せよ。



以下のような結果が
\begin{itemize}
	{\ttfamily \item float} ← 浮動小数点。実数を表す。
	{\ttfamily \item int} ← 整数
	{\ttfamily \item str} ← 文字列
\end{itemize}

なお、オブジェクト指向と呼ばれる現代的なプログラミング技法では、
型の設計が開発の中心的役割を担う。

オブジェクト指向プログラミングについては、本授業の最後に少し触れる。


と表示されたはずである。
なお、「←」の右側はそれぞれの意味を表す。

次に、以下を実行せよ。

\begin{itemize}
	{\ttfamily 
		\item d = [1.0, 2.0, 3.0]
		
		print(type(d))
		
		\item print(d[0], d[1], d[2])
				
		\item print(type(d[0]))
		
		print(type(c))
	}
\end{itemize}




\subsection{}
